% Klotski.tex

%%%%%%%%%%%%%%%%%%%%
\begin{frame}
  \begin{exampleblock}{Klotski Puzzle (华容道)}
    \fig{width = 0.45\textwidth}{figs/Klotski}
  \end{exampleblock}
\end{frame}
%%%%%%%%%%%%%%%%%%%%

%%%%%%%%%%%%%%%%%%%%
\begin{frame}
  \begin{exampleblock}{Klotski Puzzle (华容道; 中国版本)}
    \fig{width = 0.40\textwidth}{figs/HuaRongDao}
  \end{exampleblock}
\end{frame}
%%%%%%%%%%%%%%%%%%%%

%%%%%%%%%%%%%%%%%%%%
\begin{frame}
  \begin{exampleblock}{\href{https://youtu.be/YI1WqYKHi78}{15 Puzzle (数字华容道)}}
    \begin{columns}
      \column{0.50\textwidth}
        \fig{width = 0.70\textwidth}{figs/15-puzzle-wiki}
      \column{0.50\textwidth}
        \fig{width = 0.70\textwidth}{figs/15-puzzle-init}
    \end{columns}
  \end{exampleblock}
\end{frame}
%%%%%%%%%%%%%%%%%%%%

%%%%%%%%%%%%%%%%%%%%
\begin{frame}{}
  \begin{center}
    \fig{width = 0.35\textwidth}{figs/15-puzzle-unsolvable}

    \vspace{0.50cm}
    \red{\large Is it solvable?}
  \end{center}
\end{frame}
%%%%%%%%%%%%%%%%%%%%

%%%%%%%%%%%%%%%%%%%%
\begin{frame}{}
  \begin{center}
    \red{\large How to solve it?}

    \pause
    \vspace{0.80cm}
    It uses \blue{\bf permutation groups} in \blue{\bf group theory}.

    \fig{width = 0.50\textwidth}{figs/stay-tuned}
  \end{center}
\end{frame}
%%%%%%%%%%%%%%%%%%%%