% axiom.tex

%%%%%%%%%%%%%%%
\begin{frame}{}
  \begin{center}
    Self-contained (自包含; 自给自足)

    \fig{width = 0.50\textwidth}{figs/axiomatic-systems}

    Syntax {\it vs.} Semantics (语法与语义对立统一)
  \end{center}
\end{frame}
%%%%%%%%%%%%%%%

%%%%%%%%%%%%%%%
\begin{frame}{}
  \begin{center}
    {\bf \large \red{三个公理系统:} 逻辑、集合论、\gray{图论}、抽象代数(群论)}
  \end{center}
\end{frame}
%%%%%%%%%%%%%%%

%%%%%%%%%%%%%%%
% \begin{frame}{}
%   \begin{center}
%     {\Large Axiomatic Set Theory \blue{(ZFC)}}
%   \end{center}
%
%   \vspace{0.80cm}
%   \begin{columns}
%     \column{0.45\textwidth}
%       \fig{width = 0.50\textwidth}{figs/Zermelo}{\centerline{Ernst Zermelo (1871--1953)}}
%     \column{0.45\textwidth}
%       \fig{width = 0.48\textwidth}{figs/Fraenkel}{\centerline{Abraham Fraenkel (1891--1965)}}
%   \end{columns}
% \end{frame}
%%%%%%%%%%%%%%%

%%%%%%%%%%%%%%%
\begin{frame}{}
  \begin{columns}
    \column{0.50\textwidth}
      \fig{width = 0.55\textwidth}{figs/elements-ch}
    \column{0.50\textwidth}
      \fig{width = 0.50\textwidth}{figs/hilbert-geometry}
  \end{columns}

  \vspace{0.30cm}
  \begin{enumerate}[(1)]
    \item To draw a straight \blue{line} from any \blue{point} to any point.
    \item To extend a finite straight line continuously in a straight line.
    \item To describe a circle with any center and radius.
    \item That all right angles are equal to one another.
    \item \red{The parallel postulate}.
  \end{enumerate}
\end{frame}
%%%%%%%%%%%%%%%

%%%%%%%%%%%%%%%%%%%%
\begin{frame}{}
  \begin{exampleblock}{Axiomatic System for a \red{Four-point Geometry}}
    \blue{\bf \textit{Undefined terms:}} point, line, is on

    \vspace{0.50cm}
    \purple{\bf \textit{Axioms:}}
    \begin{enumerate}[(1)]
      \item There are exactly four \teal{points}.
      \item It is impossible for three \teal{points} to \teal{be on} the same line.
      \item For every pair of distinct \teal{points} $x$ and $y$,
        there is a unique \teal{line} $l$ such that $x$ \teal{is on} $l$
        and $y$ \teal{is on} $l$.
      \item Given a \teal{line} $l$ and a \teal{point} $x$ that is not \teal{on} $l$,
        there is a unique \teal{line} $m$ such that $x$ \teal{is on} $m$
        and no \teal{point} on $l$ is also \teal{on} $m$.
    \end{enumerate}
  \end{exampleblock}

  \pause
  \vspace{0.30cm}
  \begin{theorem}
    There are at least two distinct lines.
  \end{theorem}
\end{frame}
%%%%%%%%%%%%%%%

%%%%%%%%%%%%%%%%%%%%
\begin{frame}{}
  \begin{center}
    Syntax {\it vs.} Semantics

    \pause
    \vspace{0.30cm}
    \fig{width = 0.70\textwidth}{figs/point-loop}
    \[
      \text{point}: \red{\cdot} \qquad
      \text{line}: \red{\bigcirc} \qquad
      \text{is on}: \red{\bigodot}
    \]
  \end{center}
\end{frame}
%%%%%%%%%%%%%%%

%%%%%%%%%%%%%%%%%%%%
\begin{frame}{}
  \begin{center}
    \red{\bf \large 什么样的推理是正确的?}

    \fig{width = 0.70\textwidth}{figs/euler-logic}
  \end{center}
\end{frame}
%%%%%%%%%%%%%%%%%%%%

%%%%%%%%%%%%%%%%%%%%
\begin{frame}
  \begin{center}
    \fig{width = 0.60\textwidth}{figs/see-you-next-week}
  \end{center}
\end{frame}
%%%%%%%%%%%%%%%%%%%%