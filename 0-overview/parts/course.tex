% course.tex

%%%%%%%%%%%%%%%%%%%%
\begin{frame}{}
  \begin{center}
    {\large 分班教学 \\ (与计算机系赵建华老师)}

    \vspace{0.60cm}
    {\large \red{授课内容可能有出入}, 不影响考试与成绩分配}
  \end{center}
\end{frame}
%%%%%%%%%%%%%%%%%%%%

%%%%%%%%%%%%%%%%%%%%
\begin{frame}{}
  \begin{center}
    {\large 平时作业 {\it vs.} 期中测试 {\it vs.} 期末测试}

    \[
      3 \quad:\quad 3 \quad:\quad 4
    \]
    \[
      4 \quad:\quad 3 \quad:\quad 3
    \]

    \vspace{0.80cm}
    \blue{\large 弹性制}
  \end{center}
\end{frame}
%%%%%%%%%%%%%%%%%%%%

%%%%%%%%%%%%%%%%%%%%
\begin{frame}{}
  \begin{center}
    每周四晚上发布作业 \qquad 下周四23:55前提交作业

    \vspace{1.00cm}
    每次作业按 \blue{10} 分计算

    \vspace{0.50cm}
    \red{\bf 迟交:} 周四\red{前}向助教登记, 可延长两天, 总分 \blue{8} 分

    \vspace{0.50cm}
    \cyan{({\bf 作业助教:} 裴一凡、戴若石、肖江)}
  \end{center}
\end{frame}
%%%%%%%%%%%%%%%%%%%%

%%%%%%%%%%%%%%%%%%%%
\begin{frame}{}
  \begin{center}
    \teal{``教学立方''课程邀请码: PLD8QKTZ}

    \fig{width = 0.50\textwidth}{figs/qrcode-dm-class}
  \end{center}
\end{frame}
%%%%%%%%%%%%%%%%%%%%

%%%%%%%%%%%%%%%%%%%%
\begin{frame}{}
  \fig{width = 0.50\textwidth}{figs/yuefa}
\end{frame}
%%%%%%%%%%%%%%%%%%%%

%%%%%%%%%%%%%%%%%%%%
\begin{frame}{}
  \begin{center}
    \red{\Large \bf 非必要, 不点名}
  \end{center}
\end{frame}
%%%%%%%%%%%%%%%%%%%%

%%%%%%%%%%%%%%%%%%%%
\begin{frame}{}
  \begin{center}
    \red{\bf \Large 非必要, 不迟到}

    \pause
    \vspace{0.60cm}
    \blue{\bf \large 尽量吃早餐, 但不可以在教室吃早餐}
  \end{center}
\end{frame}
%%%%%%%%%%%%%%%%%%%%

%%%%%%%%%%%%%%%%%%%%
\begin{frame}{}
  \begin{center}
    \red{\bf \Large \xout{非必要}, 不抄袭; 一经发现, 后果严重}

    % \fig{width = 0.40\textwidth}{figs/plagiarism}

    \pause
    \vspace{1.50cm}
    \blue{\bf \Large 当次作业计零分; 总评扣十分}
  \end{center}
\end{frame}
%%%%%%%%%%%%%%%%%%%%

%%%%%%%%%%%%%%%%%%%%
% \begin{frame}{}
%   \begin{columns}
%     \column{0.40\textwidth}
%       \begin{center}
%         QQ 群号: \blue{\bf 612 868 020}
%
%         \fig{width = 0.80\textwidth}{figs/qq-group}
%       \end{center}
%     \column{0.60\textwidth}
%       \begin{description}
%         \item[\bf 作业助教:] 裴一凡、戴若石、肖江 \\[15pt]
%         % \item[\bf 技术支持:] 唐瑞泽
%       \end{description}
%   \end{columns}
% \end{frame}
%%%%%%%%%%%%%%%%%%%%

%%%%%%%%%%%%%%%%%%%%
\begin{frame}{}
  \begin{center}
    \fig{width = 0.40\textwidth}{figs/dm-structures-ch}

    \vspace{0.30cm}
    {\large 教材不重要, 听讲更重要}
  \end{center}
\end{frame}
%%%%%%%%%%%%%%%%%%%%

%%%%%%%%%%%%%%%%%%%%
\begin{frame}{}
  \begin{center}
    \fig{width = 0.50\textwidth}{figs/math-in-cs}

    \vspace{0.30cm}
    {\large 其它参考书随课程进度安排}
  \end{center}
\end{frame}
%%%%%%%%%%%%%%%%%%%%